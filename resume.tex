% The font could be set to Windows-specific Calibri by using the 'calibri' option
\documentclass[]{mcdowellcv}

% For mathematical symbols
\usepackage{amsmath}
\usepackage[fixed]{fontawesome5}
\usepackage[hidelinks]{hyperref}

% Set applicant's personal data for header
\name{Varun Ramani}
\contacts{
	(732) 672-5930 % \faIcon{phone} 
	\linebreak
	varun.ramani@gmail.com % \faIcon{envelope} 
	\linebreak
	https://varunramani.com % \faIcon{globe}
	\linebreak
	https://github.com/varun-ramani % \faIcon{github} 
	\linebreak
	https://linkedin.com/in/varun-ramani % \faIcon{linkedin} 
}

\begin{document}

	% Print the header
	\makeheader
	
	% Print the content
	\begin{cvsection}{Education}
		\begin{cvsubsection}{College Park, MD}{University of Maryland}{August 2020 -- May 2023}
			\begin{itemize}
				\item B.S. in Computer Science. Current GPA: 3.9 / 4.0
				\item Coursework: Networks; Data Science; Cryptography; Algorithms; Organization of Programming Languages; Intro
					  Systems (C/UNIX/Assembly); Object-Oriented Programming; Multivariate Calculus; Statistics I;
					  Discrete Math; Linear Algebra
			\end{itemize}
		\end{cvsubsection}
	\end{cvsection}

	\begin{cvsection}{Work Experience}
		\begin{cvsubsection}{Software Engineering Intern}{Meta (formerly Facebook)}{May 2022 -- August 2022}
			\begin{itemize}
				\item Designed and deployed improved identity matching system backed by hashed device ID data to
				      Facebook's products.
				\item Implemented and analyzed experiments assessing the effects of implementing the new system in the
					  identity matching backend and replacing the prior solution. Created new experiment framework to
					  assess effects on certain products.
				\item Worked in Hack, Python, and Meta's internal tools.
			\end{itemize}
		\end{cvsubsection}
		\begin{cvsubsection}{Undergraduate Research Assistant}{FIRE: The First-Year Innovation \& Research Experience}{August 2020 -- December 2021}
			\vspace{0.8em}
			\begin{itemize}
				\item Collaborated with peers to develop machine learning model using UNet architecture to perform
				      semantic segmentation on LIDAR data. Presented at undergraduate research summit.
			\end{itemize}
		\end{cvsubsection}
		\begin{cvsubsection}{Sensei (teacher)}{Code Ninjas Princeton}{March 2019 -- December 2019}
			\begin{itemize}
				\item Taught students aged 7--14 programming fundamentals through
					  game development courses in Scratch and JavaScript.
				\item Designed and led multi-day workshop on building NLP chatbots
				      powered by Python/IBM Watson.
				\item Supervised, mentored, and guided groups of up to 25 students at a time as they 
				      built their software.
			\end{itemize}
		\end{cvsubsection}
	\end{cvsection}

	\begin{cvsection}{Technical Skills}
		\begin{cvsubsection}{}{}{}
			\vspace{0.5em}
			\begin{itemize}
				\item \textbf{Fluent:} Python, Java, NodeJS, JavaScript, MongoDB, Fullstack Development
				\item \textbf{Some Experience:} Go, PostgreSQL, Flutter, React Native, C, Machine Learning
			\end{itemize}		
		\end{cvsubsection}
	\end{cvsection}

	\begin{cvsection}{Projects and Awards}
		\begin{cvsubsection}{Memaid}{Best Social Good Hack \& Best Use of Google Cloud}{\href{https://github.com/varun-ramani/memaid}{gh:varun-ramani/memaid}}
			\vspace{0.8em}
			\begin{itemize}
				\item Collaborated with 3 peers to integrate computer vision, speech to text, and NLP into a dementia aid.
				\item When someone introduces themselves to the user, the application memorizes their face \emph{after
				      only seeing it once} and associates it with their name. Furthermore, it listens to any subsequent
				      conversation and stores relevant highlights. 
				\item The next time the user meets this person, the application will automatically recognize them and
				      relay their name / last conversation highlights to the user through any connected headphones or earbuds.
				\item Competed against 91 other teams.
			\end{itemize}
		\end{cvsubsection}
		\begin{cvsubsection}{Maskif.ai}{Grand Prize, YHack 2020}{\href{https://github.com/varun-ramani/maskifai-server}{gh:varun-ramani/maskifai-server}}
			\begin{itemize}
				\item Collaborated with 3 peers to develop accessible computer vision-powered IoT product.
				\item Product helps businesses deal with anti-maskers during pandemic by intelligently 
				triggering connected smart lock when unmasked individual approaches door; 
				automatically unlocks door after they leave.
				\item Beat 42 competing teams for first place.
				\item Applied Python, Tensorflow, Flask, and Google Assistant SDK. 
			\end{itemize}
		\end{cvsubsection}
        \begin{cvsubsection}{Intellicity}{Top 30, PennApps 2019}{\href{https://github.com/varun-ramani/intellicity}{gh:varun-ramani/intellicity}}
			\begin{itemize}
				\item Collaborated with 3 peers to develop advanced mobile map application.
				\item Product uses crowdsourced information and computer vision to add rich, granular details
				to Google Maps; includes but is not limited to precise geolocation data for trash bins, bathrooms, 
				safety hazards, and parking spots. Helps people navigate unfamiliar
				places with absolute confidence, instantly finding anything they need. 
				\item Competed against 242 other teams.
				\item Applied Dart/Flutter, Python 3, MongoDB, and Flask.
			\end{itemize}
		\end{cvsubsection}
	\end{cvsection}


\end{document}
